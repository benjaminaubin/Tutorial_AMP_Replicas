
\newpage
\section{Replica computation}
\label{sec:replicas}


\subsection{Heuristic derivation}
	
	In this section, we present the heuristic derivation of the replica formula in the context of  of the \emph{committee machine}. This computation is necessary to properly guess the formula that can then be proven using the adaptive interpolation method.
	 The reader interested in the replica approach to neural networks and the committee machine is invited to look as well to some of the classical papers \cite{gardner1988optimal,mezard1989space,schwarze1992generalization,schwarze1993generalization,schwarze1993learning,monasson1995learning}.\\
	 
	We present here the replica computation of the averaged free entropy $\Phi(\alpha)$ in \eq\eqref{free_entropy} for arbitrary \emph{student} prior and channel distributions $\rP_{\w},\rP_{\w^\star}$ and $\rP_{\out},\rP_{\out^\star}$, so that the computation remains valid for both the Bayes-optimal and mismatched settings.\\
	

	\subsubsection{Replica trick}
	\label{sec:replict_rick}
		The average in eq.~\eqref{free_entropy} is intractable in general, and the computation relies on the so called \emph{replica trick}, that consists in applying the identity  
				\begin{align}
					\EE_{\vec{y},\mat{X}} \[ \lim_{\ndim \to \infty} \frac{1}{\ndim} \log  \mZ_\ndim\(\vec{y}, \mat{X}\) \] =  \lim_{r \to 0} \[ \lim_{\ndim \to \infty}  \frac{1}{\ndim}  \frac{\partial \log \EE_{\vec{y},\mat{X}} \[  \mZ_\ndim\(\vec{y}, \mat{X}\)^r\] }{\partial r} \]\,.
					\label{appendix:replica:committee:replica_trick}
				\end{align}
		The replica trick has been used in a series of previous works to compute the free energy density of GLM for separable distributions \cite{krzakala2012probabilistic} and has been rigorous proven in this case by \cite{barbier2017phase}.\\
		
		Eq.~\eqref{appendix:replica:committee:replica_trick} is interesting in the sense that it reduces the intractable average to the computation of the moments of the averaged partition function, which are easier quantities to compute. Note that for $r \in \bbN$, $\mZ_\ndim\(\vec{y}, \mat{X}\)^r = \prod_{a=1}^r \mZ_\ndim\(\vec{y}, \mat{X}\)$ represents the partition function of $r$ identical non-interacting copies of the initial system, called \emph{replicas}. Taking the quenched average over the disorder will then correlate the replicas, before taking the number of replicas $r\to 0$.	
		Therefore, we assume there exists an analytical continuation so that $r\in \bbR$ and the limit is well defined. Finally, notice that we exchanged the order of the limits $r \to 0$ and $\ndim \to \infty$. These technicalities are crucial points but are not rigorously justified and we will ignore them in the rest of the computation.\\
	
		First, in order to decouple the contributions of the channel $\rP_\out$ and the prior $\rP_\w$, we introduce the variable $\mat{Z} = \frac{1}{\sqrt{\ndim}} \mat{X} \mat{W}$ and a Dirac-delta integral:
			\begin{align}
				\mZ_\ndim\(\vec{y}, \mat{X}\) &= \int_{\bbR^{\nsamples\times K}} \d\vec{z} ~ \rp_{\out}\(\vec{y} | \mat{Z} \) \int_{\bbR^{\ndim\times K}} \d\vec{w} ~ \rp_{\w}\(\mat{W}\) \delta\(\mat{Z} - \frac{1}{\sqrt{\ndim}} \mat{X} \mat{W}\)\,.
			\end{align} 
			Thus the replicated partition function for an integer $r\in \bbN$ in eq.~\eqref{appendix:replica:committee:replica_trick} can be  written as
			\begin{align}
			&\EE_{\vec{y},\mat{X}} \[  \mZ_\ndim\(\vec{y}, \mat{X}\)^r \] \nonumber \\
			&=  \EE_{\vec{y},\mat{X}} \[  \prod_{a=1}^r \int_{\bbR^\nsamples} \d\mat{Z}^a ~ \rp_{\out^a}\(\vec{y} | \mat{Z}^a \) \right. \\
			& \qquad \qquad \left. \times  \int_{\bbR^{\ndim \times K}} \d\mat{W}^a ~ \rp_{\w^a}\(\mat{W}^a\) \delta\(\mat{Z}^a - \frac{1}{\sqrt{\ndim}}\mat{X} \mat{W}^a\)\] \nonumber \\ 
			&= \EE_{\mat{X}} \int_{\bbR^\nsamples} \d \vec{y} ~ \int_{\bbR^{\nsamples \times K}} \d\mat{Z}^\star ~ \rp_{\out^\star}\(\vec{y} | \mat{Z}^\star \)  \\
			& \qquad \qquad \times  \int_{\bbR^{\ndim \times K}} \d\mat{W}^\star ~ \rp_{\w^\star}\(\mat{W}^\star\) \delta\(\mat{Z}^\star - \frac{1}{\sqrt{\ndim}}\mat{X} \mat{W}^\star\)\nonumber \\
			& \times \[  \prod_{a=1}^r \int_{\bbR^{\nsamples \times K}} \d\mat{Z}^a ~ \rp_{\out^a}\(\vec{y} | \mat{Z}^a \) \int_{\bbR^{\ndim\times K}} \d\mat{W}^a ~ \rp_{\w^a}\(\mat{W}^a\) \delta\(\mat{Z}^a - \frac{1}{\sqrt{\ndim}}\mat{X} \mat{W}^a\)\]  \nonumber \\
			&= \int_{\bbR^\nsamples} \d \vec{y} ~ \prod_{a=0}^r \int_{\bbR^{\nsamples \times K}} \d\mat{Z}^a ~ \rp_{\out^a}\(\vec{y} | \mat{Z}^a \) \int_{\bbR^{\ndim \times K}} \d\mat{W}^a ~ \rp_{\w^a}\(\mat{W}^a\) \\
			& \qquad \qquad  \times \underbrace{\EE_{\mat{X}} \prod_{a=0}^r \delta\(\mat{Z}^a - \frac{1}{\sqrt{\ndim}}\mat{X} \mat{W}^a\)}_{(I)}\,. \nonumber
			\end{align}
			Note that the average over $\vec{y}$ is equivalent to the one over the ground truth vector $\mat{W}^\star$ in the case of a \emph{teacher-student}, which can be conveniently grouped with the other terms by just extending the replica indices and
			considering it as a new replica $\mat{W}^0$ with index $a=0$, leading to a total of $r+1$ replicas. 
			
		%%%% Average %%%%
	\subsubsection{Average over the i.i.d input data $\mat{X}$}	
	\label{average_data}	
			Remains to compute the average over $\mat{X}$ in the term $(I)$.
			We suppose that inputs are drawn from an \iid distribution, for example a Gaussian $\rP_{\textrm{x}}(\vec{x}) = \mN_{\vec{x}}\(\vec{0},\mat{I}_\ndim\)$. More precisely, for $(i,j) \in \lb \ndim\rb^2$, $(\mu,\nu) \in \lb \nsamples \rb^2$, $\EE_\mat{X} \[ x_{\mu i} x_{\nu j} \] =  \delta_{\mu\nu} \delta_{ij}$. \\
			By definition, the average in $(I)$ defines the probability density $\rp_{\z^a} (\mat{Z}^a)$ and as $\forall k \in \lb K \rb, \forall \mu \in \lb\nsamples \rb$, $z_{\mu k}^a =\frac{1}{\sqrt{\ndim}} \sum_{i=1}^\ndim x_{\mu i} w_{ik}^a$ is the sum of \iid random variables, the CLT insures that in the thermodynamic limit $\ndim \to \infty$, $z_{\mu k}^a$ follows a Gaussian multivariate distribution, with first moments given by:
			\begin{align}
					\EE_{\mat{X}}[z_{\mu k}^a] &= \frac{1}{\sqrt{\ndim}} \sum_{i=1}^\ndim \EE_{\mat{X}}\[x_{\mu i}\] w_{ik}^a = 0 \spacecase
					\EE_{\mat{X}}[z_{\mu k}^a z_{\nu k'}^b] &= \frac{1}{\ndim} \sum_{ij} \EE_{\mat{X}}\[x_{\mu i} x_{\mu j}\] w_{ik}^a w_{jk'}^b  = \frac{1}{\ndim} \sum_{ij}  \delta_{ij} w_{ik}^a w_{jk'}^b \delta_{\mu \nu} \equiv \delta_{\mu \nu} ~ Q^{a k}_{b k'}. \nonumber
			\end{align}
			
			Notice that averaging over the quenched disorder introduced correlations between replicas, which were initially independent, described by the symmetric \emph{overlap} matrix $\{ Q^{a k}_{b k'}\}_{kk'}$ of size $(r+1)K \times (r+1)K$.
			This matrix order parameter measures the correlations between the replicated matrices $\{\mat{W}^a\}_{a=0}^r$ and is formally defined by
			$$\mat{Q}(\{\mat{W}^a\}_{a=0}^r)\equiv\(\frac{1}{\ndim} \sum_{i=1}^{\ndim} w_{ik}^a w_{ik'}^b \)^{a,b=0..r}_{k,k'=1..K}\,,$$ 
			such that $\forall (a,b) \in \lb 0 : r \rb^2 $, $\mat{Q}^{ab} \in \bbR^{K \times K}$. 
			Therefore, again by the CLT, in the limit $d\to\infty$, the hidden variable $\mat{Z}^a \in \bbR^{\nsamples \times K}$ converges in distribution to the multivariate distribution
			\begin{align}
				\rP_{\z^a}\(\mat{Z}^a|\mat{Q}\) = \exp \left[- \frac{1}{2} \sum_{\mu =1}^\nsamples \sum_{a,b=0}^r \sum_{k,k'=1}^K z^{a}_{\mu k} z^b_{\mu k'}(\mat{Q}^{-1})_{k k'}^{ab} \right] / \( \det{2\pi \mat{Q}} \)^{\frac{\nsamples}{2}}\,.
			\end{align}
				
			Inserting this back in the replicated partition function finally writes
			
			\begin{align}
			&\EE_{\vec{y},\mat{X}} \[  \mZ_\ndim\(\vec{y}, \mat{X}\)^r \] = \nonumber \\
			& \qquad \int_{\bbR^\nsamples} \d \vec{y} ~ \prod_{a=0}^r \int_{\bbR^{\nsamples \times K}} \d\mat{Z}^a ~ \rp_{\out^a}\(\vec{y} | \mat{Z}^a \) \rp_{\z^a}\(\mat{Z}^a|\mat{Q}\) \int_{\bbR^{\ndim \times K}} \d\mat{W}^a ~ \rp_{\w^a}\(\mat{W}^a\)
			\end{align}
						
						
	\subsubsection{Fourier representation}
			Next we introduce the change of variable for the new order parameter $\mat{Q}^{ab}$ with a Dirac-$\delta$ distribution and its Fourier representation. For a variable $x \in \bbR$, the distribution $\delta(x)$ can be written as an integral over a purely imaginary parameter $\hat{x}$:
			\begin{align*}
				\delta\(x\) = \frac{1}{2 i\pi} \int_{i \bbR} \d \hat{x} e^{-\hat{x} x }\,.
			\end{align*}
			
			 Applying the above identity to the change of variable, we obtain
			 \begin{align}
			 \begin{aligned}
			 	&1 = \int_{\bbR^{(K \times r+1)^2 }} \d \mat{Q} \prod_{0 \leq a \leq b \leq r, 1 \leq k,k'\leq K} \delta \(\ndim Q^{ab}_{kk'}-\sum_{i=1}^\ndim w_{ik}^a w_{ik'}^b \)\\
				&\propto \int \int_{\bbR^{(K \times r+1)^2}} \d \mat{Q} \d\hat{\mat{Q}} ~ \exp \( - \ndim \sum_{a=0}^r \sum_{k, k'}^K Q^{aa}_{k k'} \hat{Q}^{aa}_{k k'} - \frac{\ndim}{2} \sum_{a \neq b}^r \sum_{k, k'}^K Q^{a b}_{k k'} \hat{Q}^{a b}_{k k'}  \) \\
				&  \qquad \qquad \times \exp \(\sum_{a=0}^r \sum_{k, k'}^K  \hat{Q}^{a a}_{k k'} w_k^a w_{k'}^a  + \frac{1}{2} \sum_{a \neq b}^r \sum_{k,k'}^K \hat{Q}^{a b}_{k k'} w_k^a w_{k'}^b\) \,,
			\end{aligned}
			\end{align} 
			that involves a new ad-hoc purely imaginary matrix parameter $\hat{\mat{Q}} \in i\bbR^{(K \times (r+1))^2 }$.
			Finally, multiplying the replicated partition function by $1$, using the Cauchy theorem and rotating the integration, it becomes an integral over the symmetric matrices $\mat{Q} \in \bbR^{(K \times r+1)^2}$ and $\hat{\mat{Q}} \in \bbR^{(K \times r+1)^2}$
			 \begin{align*}       
			  &\EE_{\vec{y},\mat{X}} \[  \mZ_\ndim\(\vec{y}, \mat{X}\)^r \] \\
			  &= \int \int_{\bbR^{(K \times r+1)^2}} \d \mat{Q} \d\hat{\mat{Q}} ~ \exp \( - \ndim \sum_{a=0}^r \sum_{k,k'}^K Q^{aa}_{kk'} \hat{Q}^{aa}_{kk'} - \frac{\ndim}{2} \sum_{a \neq b}^r \sum_{k, k'}^K Q^{a b}_{k k'} \hat{Q}^{a b}_{k k'}  \) \\
			  & \qquad \qquad \times \exp \( \sum_{a=0}^r \sum_{k, k'}  \hat{Q}^{a a}_{k k'} w_k^a w_{k'}^a  + \frac{1}{2} \sum_{a \neq b} \sum_{k, k'} \hat{Q}^{a b}_{k k'} w_k^a w_{k'}^b\)\\
			  & \int_{\bbR^\nsamples} \d \vec{y}  \prod_{a=0}^r \int_{\bbR^{\nsamples \times K}} \d\mat{Z}^a ~ \rp_{\out^a}\(\vec{y} | \mat{Z}^a \) \rp_{\z^a}\(\mat{Z}^a | \mat{Q} \) \int_{\bbR^{\ndim \times K}} \d\mat{W}^a ~ \rp_{\w^a}\(\mat{W}^a\) \\
			  &= \int \int_{\bbR^{(K \times r+1)^2}} \d \mat{Q} \d\hat{\mat{Q}} ~ \exp \( - \ndim \sum_{a=0}^r \sum_{k, k'}^K Q^{aa}_{kk'} \hat{Q}^{aa}_{kk'} - \frac{\ndim}{2} \sum_{a \neq b}^r \sum_{k, k'}^K Q^{a b}_{k k'} \hat{Q}^{a b}_{k k'}  \) \\
			  & \qquad \qquad \times \exp \( \sum_{a=0}^r \sum_{k, k'}^K  \hat{Q}^{a a}_{k k'} w_k^a w_{k'}^a  + \frac{1}{2} \sum_{a \neq b}^r \sum_{k,k'}^K \hat{Q}^{a b}_{k k'} w_k^a w_{k'}^b\)\\
			  &  \[\int_\bbR \d y  \prod_{a=0}^r \int_{\bbR^{K}} \d\vec{z}^a ~ \rp_{\out^a}\(y | \vec{z}^a \) \rp_{\z^a}\(\vec{z}^a | \mat{Q} \)\]^\nsamples \[\prod_{a=0}^r \int_{\bbR^{K}} \d\vec{w}^a ~ \rp_{\w^a}\(\vec{w}^a\)\]^{\ndim} \\
			  &\simeq  \int \int_{\bbR^{(K \times r+1)^2}} \d \mat{Q} \d\hat{\mat{Q}} ~ e^{\ndim \Phi^{(r)} (\mat{Q},\hat{\mat{Q}} )} \,,
			\end{align*}    
			where in the last step, we used a Laplace method \cite{Rong89} and omitted the sub-leading factors in the thermodynamic limit $\ndim \to \infty$ to evaluate it as a function of the free entropy potential defined by
			\begin{align}
			\begin{aligned}
				\Phi^{(r)} (\mat{Q},\hat{\mat{Q}}) &\equiv  - \sum_{a=0}^r \sum_{k, k'}^K Q^{a a}_{k k'} \hat{Q}^{a a}_{k k'} -\frac{1}{2}\sum_{a \neq b}^r \sum_{k, k'}^K Q^{a b}_{k k'} \hat{Q}^{a b}_{k k'} \\
				&+ \log \Psi_{\w}^{(r)} (\hat{\mat{Q}})  + \alpha \log \Psi_{\out}^{(r)}(\mat{Q})\,,
			      \spacecase
			      \Psi_{\w}^{(r)} (\hat{\mat{Q}}) &\equiv \displaystyle \prod_{a=0}^r \int_{\bbR^{K}} \d \vec{w}^a ~ 
			      \rp_{\w^a}\(\vec{w}^a\) \\
			      & \hhspace \times \exp \( \sum_{a=0}^r \sum_{k k'}^{K}  \hat{Q}^{a a}_{k k'} w_k^a w_{k'}^a  + \frac{1}{2} \sum_{a \neq b}^r \sum_{k,k'}^{K} \hat{Q}^{a b}_{k k'} w_k^a w_{k'}^b\)\,, \spacecase
			      \Psi_{\out}^{(r)}(\mat{Q}) &\equiv \displaystyle \prod_{a=0}^r  \int_\bbR \d y \int_{\bbR^{K}}  \d \vec{z}^a ~ \rp_{\out^a}\(y | \vec{z}^a\) \rp_{\z^a}\(\vec{z}^a | \mat{Q}\)\,,
			\end{aligned}
			    \label{appendix:intro:replicas:Phi_r}
			\end{align}
			and where we decoupled the variable $\Z^a \in \bbR^{\nsamples \times K}$ and $\mW^a \in \bbR^{\ndim \times K}$ along the rows
			\begin{align*}
				\rp_{\out^a}\(\vec{y} | \mat{Z}^a \) &= \displaystyle \prod_{\mu=1}^\nsamples \rp_{\out^a}\(y_\mu | \vec{z}^a_\mu \)\,, \text{ with } \vec{z}^a_\mu \in \bbR^{K}\,, \\
				\rp_{\z^a} (\mat{Z}^a| \mat{Q}) &= \displaystyle \prod_{\mu=1}^\nsamples \rp(\vec{z}^a_\mu | \mat{Q})\,, \\
				\rp_{\w^a}\(\mat{W}^a\) &= \displaystyle \prod_{i=1}^\ndim \rp_{\w^a}\( \vec{w}^a_i \)\,, \text{ with } \vec{w}^a_i \in \bbR^{K}\,,\\
				\rp_{\z^a}\(\vec{z}^a | \mat{Q}\) &= \exp \left[- \frac{1}{2} \sum_{a,b=0}^r \sum_{k,k'=1}^K z^{a}_{k} z^b_{k'}(\mat{Q}^{-1})_{k k'}^{ab} \right] / \( \det{2\pi \mat{Q}} \)^{\frac{1}{2}}\,.
			\end{align*}
			Note that the averaged replicated partition function of this fully connected model can be expressed as a saddle point equation only because distributions $\rP_{\out},\rP_{\out^\star}$ and $\rP_\w,\rP_{\w^\star}$ are separable so that a pre-factor scaling with the system size $\ndim$ dominates the exponential distribution.
			Finally, switching the two limits $r\to 0$ and $\ndim \to \infty$, the quenched free entropy $\Phi$ simplifies as a saddle point equation
			\begin{equation}
				\Phi (\alpha) = \extr_{ \mat{Q}, \hat{\mat{Q}} } \left\{\lim_{r\rightarrow 0} \frac{\partial \Phi^{(r)}(\mat{Q},\hat{\mat{Q}})}{\partial  r} \right\}\,,
			\end{equation}
			over symmetric matrices $\mat{Q}\in \bbR^{(K \times r+1)^2}$ and $\hat{\mat{Q}} \in \bbR^{(K \times r+1)^2}$.\\
			 
			To summarize, we managed to get rid of the original high-dimensional integrals and replace them by an optimization in the space of matrices, which, in this form, is still intractable. We not only have to search in the space of $(r+1)\times (r+1)$ matrices to find the extremiser of $\Phi^{(r)}$, but we also need to compute the limit $r\to 0^{+}$.
			In the following we will assume a simple Ansatz for these matrices in order to first obtain an analytic expression in $r$ before taking the derivative with respect to $r$.

		\subsection{Replica Symmetric free entropy}
		\label{rs_free_entropy}			
			Our goal is to express the functional $\Phi^{(r)}(\mat{Q},\hat{\mat{Q}})$ appearing in the free entropy as an analytical function of $r$, in order to perform the replica trick. 
			
			\subsubsection{Replica symmetric ansatz}
			To do so, we will assume that the extremum of $\Phi^{(r)}$ is attained at a point in $\mat{Q},\hat{\mat{Q}}$ space such that a \emph{replica symmetry} property is verified. More concretely, we assume: 
			\begin{align}
			\begin{aligned}
			\exists \mat{Q} \in \bbR^{K \times K} \text{ s.t } \quad \forall a \in \lb 0 : r \rb \quad \forall (k,k') \in \lb K \rb^2 \quad Q^{a a}_{k k'} &= Q_{k k'}\,,\\
			\exists \mat{Q}^\star \in \bbR^{K \times K} \text{ s.t } \quad \forall (k,k') \in \lb K \rb^2 \quad Q^{0 0}_{k k'} &= Q^\star_{k k'}\,,\\
			\exists \mat{q} \in \bbR^{K \times K} \text{ s.t } \quad \forall (a < b) \in \lb 0 : r \rb^2 \quad \forall (k,k') \in \lb K \rb^2 \quad Q^{a b}_{k k'} &= q_{k k'}\,,\\
			\exists \mat{m} \in \bbR^{K \times K} \text{ s.t } \quad \forall a \in \lb 0 : r\rb \quad \forall (k,k') \in \lb K \rb^2 \quad Q^{0 a}_{k k'} &= m_{k k'}\,,
			\end{aligned}
			\end{align}
			and similarly for the ad-hoc parameter
			\begin{align}
			\begin{aligned}
			\exists \hat{\mat{Q}} \in \bbR^{K \times K} \text{ s.t } \quad \forall a \in \lb 0 : r \rb \quad \forall (k,k') \in \lb K \rb^2 \quad \hat{Q}^{a a}_{k k'} &= -\frac{1}{2} \hat{Q}_{k k'}\,,\\
			\exists \hat{\mat{Q}}^\star \in \bbR^{K \times K} \text{ s.t } \quad \forall (k,k') \in \lb K \rb^2 \quad \hat{Q}^{0 0}_{k k'} &= \hat{Q}^\star_{k k'}\,, \\
			\exists \hat{\mat{q}} \in \bbR^{K \times K} \text{ s.t } \quad \forall (a < b) \in \lb 0 : r \rb^2 \quad \forall (k,k') \in \lb K \rb^2 \quad \hat{Q}^{a b}_{k k'} &= \hat{q}_{k k'}\,, \\
			\exists \hat{\mat{m}} \in \bbR^{K \times K} \text{ s.t } \quad \forall a \in \lb 0 : r \rb \quad \forall (k,k') \in \lb K \rb^2 \quad \hat{Q}^{0 a}_{k k'} &= \hat{m}_{k k'}\,.
			\end{aligned}\
			\end{align} 
			The factor $-\frac{1}{2}$ is not necessary bu useful to recover commonly used formulations.
			This Ansatz can be represented by symmetric RS matrices $\mat{Q}^{(\rs)} \in \bbR^{(K \times r+1)^2}$ and $\hat{\mat{Q}}^{(\rs)} \in \bbR^{(K \times r+1)^2}$
			\begin{equation}
			\begin{aligned}[c]
			\mat{Q}^{(\rs)} = \scalemath{0.95}{\begin{pmatrix} 
			\mat{Q}^\star & \mat{m} & \cdots & \mat{m} \\
			\mat{m}^\intercal  & \mat{Q}  & \mat{q} & ...  \\
			\vdots & \mat{q} & \ddots & \mat{q}   \\
			\mat{m}^\intercal  &... & \mat{q}  & \mat{Q}     \\
			\end{pmatrix}}
			\end{aligned}
			\hspace{0.2cm}
			\textrm{and} 
			\hspace{0.2cm}
			\begin{aligned}[c]
			\hat{\mat{Q}}^{(\rs)}=\scalemath{0.9}{\begin{pmatrix} 
			 \hat{\mat{Q}}^\star & \hat{\mat{m}} & ... & \hat{\mat{m}}\\
			\hat{\mat{m}}^\intercal &-\frac{1}{2}\hat{\mat{Q}} & \hat{\mat{q}} & ...  \\
			 \vdots & \hat{\mat{q}} & \ddots & \hat{\mat{q}}  \\
			\hat{\mat{m}}^\intercal &... & \hat{\mat{q}} & -\frac{1}{2}\hat{\mat{Q}}\\  
			\end{pmatrix}}\,,
			\end{aligned}
			\end{equation}
			where the \emph{overlap} parameters may be reinterpreted as the scalar product between the replicas
			\begin{align*}
				\forall (a,b) \in \lb r \rb^2,~ \mat{q} = \frac{1}{\ndim} \mat{W}^{a \intercal} \mat{W}^{b}\,,
			\end{align*}
			the self-overlap of each replica
			\begin{align*}
				\forall a \in \lb r \rb,~ \mat{Q} = \frac{1}{\ndim} \mat{W}^{a \intercal} \mat{W}^{a}\,,
			\end{align*}
			the scalar product with the ground truth
			\begin{align*}
				\forall a \in \lb r \rb,~ \vec{m} = \frac{1}{\ndim} \mat{W}^{\star \intercal} \mat{W}^{a}\,,
			\end{align*}
			and the second moment of the ground truth distribution
			\begin{align*}
				\mat{Q}^\star = \frac{1}{\ndim} \mat{W}^{\star \intercal} \mat{W}^{\star}\,.
			\end{align*}
			The above Ansatz simplifies in the scalar GLM case with $K=1$ to
			$q = \frac{1}{\ndim} \vec{w}^a \cdot \vec{w}^b$ for $a \ne b$, a norm $Q= \frac{1}{\ndim} \|\vec{w}^a\|_2^2$, an overlap with the ground truth $m =\frac{1}{\ndim} \vec{w}^a \cdot \vec{w}^\star$ and a second moment $Q^\star= \frac{1}{\ndim} \|\vec{w}^\star\|_2^2$.\\
		
		\subsubsection{Decomposition of the free entropy functional}
			Let's compute separately the terms involved in the functional $\Phi^{(r)}(\mat{Q},\hat{\mat{Q}})$ in \eqref{appendix:intro:replicas:Phi_r} by applying this Ansatz: the first is a trace term, the second term $\Psi_{\w}^{(r)}$ depends on the prior distributions $\rP_\w$, $\rP_{\w^\star}$ and finally the third term $\Psi_{\out}^{(r)}$ depends on the channel distributions $\rP_{\out^\star}$, $\rP_\out$.
						
		\paragraph{Trace term} 
				The trace term in \eqref{appendix:intro:replicas:Phi_r} can be easily computed at the RS fixed point and takes the following form
				\begin{align}
				\begin{aligned}
					&\left. - \sum_{a=0}^r \sum_{k, k'}^K Q^{a a}_{k k'} \hat{Q}^{a a}_{k k'} -\frac{1}{2}\sum_{a \neq b}^r \sum_{k, k'}^K Q^{a b}_{k k'} \hat{Q}^{a b}_{k k'} \right|_{\rs} \\
					&= - \tr{\mat{Q}^\star \hat{\mat{Q}}^\star} + \frac{1}{2} r \tr{\mat{Q} \hat{\mat{Q}}} -  r \tr{\mat{m} \hat{\mat{m}}}  - \frac{r(r-1)}{2} \tr{\mat{q} \hat{\mat{q}}}\,,
				\end{aligned}
				\end{align}
				and taking the derivative and the limit $r\to 0$ we obtain
				\begin{align}
				\begin{aligned}
					&\lim_{r \to 0} \partial_r \( \left. - \sum_{a=0}^r \sum_{k, k'}^K Q^{a a}_{k k'} \hat{Q}^{a a}_{k k'} -\frac{1}{2}\sum_{a \neq b}^r \sum_{k, k'}^K Q^{a b}_{k k'} \hat{Q}^{a b}_{k k'} \)\right|_{\rs} \\
					&  \qquad \qquad = \frac{1}{2} \tr{\mat{Q} \hat{\mat{Q}}} -  \tr{\mat{m} \hat{\mat{m}}}  + \frac{1}{2} \tr{\mat{q} \hat{\mat{q}}}
					\label{appendix:replicas:committee:trace} 
				\end{aligned}
				\end{align}
			
			\paragraph{Prior integral $\Psi_{\w}^{(r)}$} Evaluated at the RS fixed point the quadratic form reads			      
				\begin{align*}
					&\sum_{a=0}^r \sum_{k k'}  \hat{Q}^{a a}_{k k'} w_k^a w_{k'}^a  + \frac{1}{2} \sum_{a \neq b} \sum_{k,k'} \hat{Q}^{a b}_{k k'} w_k^a w_{k'}^b\\
					&= \vec{w}^{\star \intercal} \hat{\mat{Q}}^\star \vec{w}^{\star} + \sum_{a=1}^r \vec{w}^{\star \intercal} \hat{\mat{m}} \vec{w}^a - \frac{1}{2} \sum_{a=1}^r  \vec{w}^{a \intercal} \hat{\mat{Q}} \vec{w}^{a} + \frac{1}{2} \sum_{1 \leq a \ne b \leq r}  \vec{w}^{a \intercal} \hat{\mat{q}} \vec{w}^{b} \\
					&= \vec{w}^{\star \intercal} \hat{\mat{Q}}^\star \vec{w}^{\star} + \sum_{a=1}^r \vec{w}^{\star \intercal} \hat{\mat{m}} \vec{w}^a - \frac{1}{2} \sum_{a=1}^r  \vec{w}^{a \intercal} \(\hat{\mat{Q}} + \hat{\mat{q}} \) \vec{w}^{a} + \frac{1}{2} \(\sum_{a=1}^r  \vec{w}^{a} \)^\intercal \hat{\mat{q}} \(\sum_{a=1}^r  \vec{w}^{a} \)\,.
				\end{align*}
				
				Using a Hubbard-Stratonovich transformation:
				\begin{proposition}[Hubbard-Stratonovich transformation]
				\label{prop:hubbard}
					 For $\bxi \sim \mN\(\vec{0},\rI_\ndim\)$ and a symmetric positive definite matrix $\mat{A} \in \bbR^{\ndim \times \ndim}$, for all $\vec{x} \in \bbR^{\ndim}$
					 \begin{align}
					\EE_{\bxi}\exp\(  \bxi^\intercal \mat{A}^{1/2} \vec{x} \) = \(2\pi\)^{-\ndim/2} \int_{\bbR^\ndim} \d \bxi e^{-\frac{1}{2} \vec{x}^\intercal \vec{x} + \bxi^\intercal \mat{A}^{1/2} \vec{x} } = e^{\frac{1}{2} \vec{x}^\intercal \mat{A} \vec{x} }\,,
					\end{align}
				\end{proposition}
				
				the prior integral can be further simplified 
				
				\begin{align}
				\begin{aligned}
					&\left.\Psi_{\w}^{(r)} (\hat{\mat{Q}})\right|_{\rs} = \displaystyle \int_{\bbR^{(r+1) \times K}} \d \mat{W} ~ 
			        \rp_{\w}\(\mat{W}\) e^{\sum_{a=0}^r \sum_{k, k'} ^K \hat{Q}^{a a}_{k k'} w_k^a w_{k'}^a  + \frac{1}{2} \sum_{a \neq b} \sum_{k,k'}^K \hat{Q}^{a b}_{k k'} w_k^a w_{k'}^b} \\
					&= \EE_{\bxi, \vec{w}^\star \sim \rP_{\w^\star}} \[ e^{\vec{w}^{\star \intercal} \hat{\mat{Q}}^\star \vec{w}^{\star} } \EE_{\vec{w} \sim \rP_{\w}} \[ e^{\( \vec{w}^\intercal \hat{\vec{m}} \vec{w}^\star  - \frac{1}{2}\vec{w}^\intercal  (\hat{\mat{Q}} + \hat{\mat{q}}) \vec{w} + \vec{w}^\intercal \hat{\mat{q}}^{1/2} \bxi \) } \]^r   \]\,.
				\end{aligned}
				\label{appendix:replicas:committee:Psi_w_rs}
				\end{align}
				
			\paragraph{Detailed computation of $\(\mat{Q}^{(\rs)}\)^{-1}$}
				Let us focus on the matrix $\mat{Q}^{(\rs)}$ involved in the expression of $\Psi_{\out}^{(r)}$ in \eqref{appendix:intro:replicas:Phi_r}.
				Given the replica symmetric ansatz, we must compute the determinant and inverse matrix of the matrix $\bSigma \equiv \mat{Q}^{(\rs)}$.
				
				We use tensor products notations $\otimes$ for matrices living in $\bbR^{ r \times r } \otimes  \bbR^{ K \times K }$. Let us recall some basic properties of the tensor product of two matrices $\mat{A} \in \bbR^{r \times r}$ and $\mat{B} \in \bbR^{K \times K}$:
			\begin{align}
				(\mat{A} \otimes \mat{B})^{-1} &= \mat{A}^{-1} \otimes \mat{B}^{-1} \\
				\det{\mat{A} \otimes \mat{B}} &= \det{\mat{A}}^K \det{\mat{B}}^r \\
				\forall \mat{C}, \mat{D} \quad (\mat{C} \otimes \mat{D}) (\mat{A} \otimes \mat{B}) &= (\mat{C} \mat{A}) \otimes (\mat{D} \mat{B}) \\
				\text{Sp}(\mat{A} \otimes \mat{B}) &= \{\lambda_i \mu_j \quad  / \lambda_i \in \text{Sp}(\mat{A}),\mu_j\in \text{Sp}(\mat{B}) \}
			\end{align}
			
			Next we denote the \emph{reduced} matrix 
			\begin{align}
				\bSigma_r \equiv \begin{pmatrix}
				 \mat{Q} & \mat{q} & \cdots & \mat{q} \\
				 \mat{q} & \mat{Q} & \cdots & \mat{q} \\
				 \vdots & \vdots & \ddots & \vdots \\
				 \mat{q} & \mat{q} & \cdots & \mat{Q} \\
				\end{pmatrix} =  \mat{I}_r \otimes (\mat{Q} - \mat{q}) + \mat{J}_r \otimes \mat{q}\,,
			\end{align}
			where $\mat{J}_r$ is the rank-1 matrix with all elements equal to 1. Finally, some basic linear algebra formulas allow us to reduce the problem of computing $\det{\bSigma}$ and $\bSigma^{-1}$ to those of $\bSigma_r$. 
			
		\paragraph{Calculation of $\mat{S}$} 
			We define the $K \times K$ matrix :
			\begin{align}
				\mat{S} \equiv \mat{Q}^\star - (\vec{1}_r^\intercal
				\otimes \mat{m}) \bSigma_r^{-1} (\vec{1}_r
				\otimes \mat{m}^\intercal)  
			\end{align}
			
			Let us compute $\mat{S}$: 
			\begin{align}
				\begin{aligned}
				\mat{S} &\equiv \mat{Q}^\star - (\vec{1}_r^\intercal
				\otimes \mat{m}) \bSigma_r^{-1} (\vec{1}_r
				\otimes \mat{m}^\intercal)   \nonumber \\
				&= \mat{Q}^\star - r \mat{m} (\mat{Q} - \mat{q})^{-1} \mat{m}^\intercal + r^2 \mat{m} (\mat{Q} + (r-1)\mat{q})^{-1} \mat{q} (\mat{Q} - \mat{q})^{-1} \mat{m}^\intercal \nonumber \\
				&= \mat{Q}^\star - r \mat{m}  \(\mat{I}_K - n (\mat{Q} + (r-1)\mat{q})^{-1}  \mat{q} \)(\mat{Q} - \mat{q})^{-1} \mat{m}^\intercal \nonumber \\
				&= \mat{Q}^\star - r \mat{m} (\mat{Q} + (r-1)\mat{q})^{-1} \(\mat{Q} + (r-1)\mat{q} - r  \mat{q} \) (\mat{Q} - \mat{q})^{-1} \mat{m}^\intercal \nonumber \\
				&= \mat{Q}^\star - r \mat{m} (\mat{Q} + (r-1)\mat{q})^{-1} \mat{m}^\intercal
				\end{aligned}
			\end{align}
				
		\paragraph{Calculation of $\bSigma_r^{-1}$}
			
			$\bSigma_r$ can be easily analyzed, thanks to a slight generalization of the Sherman-Morrison formula.
			\begin{align}
				\bSigma_r =  \mat{I}_r \otimes (\mat{Q} - \mat{q}) + \mat{J}_r \otimes \mat{q} =  \mat{I}_r \otimes (\mat{Q} - \mat{q}) + \vec{1}_r \vec{1}_r^\intercal \otimes  \bq
			\end{align}
			So that the determinant reads
			\begin{align}
				\det{\bSigma_r} &= \det{\mat{Q} - \mat{q}}^r \det{\mat{I}_K + ( \vec{1}^\intercal \otimes \mat{q}) (\mat{I}_r \otimes (\mat{Q} - \mat{q}))^{-1} (\vec{1} \otimes \mat{I}_K)} \nonumber \\
				&= \det{\mat{Q} - \mat{q}}^r \det{ \mat{I}_K + r \bq (\mat{Q} - \mat{q})^{-1} } \nonumber \\
				\det{\bSigma_r} &=  \det{\mat{Q} - \mat{q}}^{r-1} \det{\mat{Q} + (r-1)\mat{q} }\,,
			\end{align}
			
			and the inverse 
			\begin{align}
			\bSigma_r^{-1} &= \mat{I}_r \otimes (\mat{Q} - \mat{q})^{-1} \\
			& - \(\mat{I}_r \otimes (\mat{Q} - \mat{q})^{-1}\) \( \vec{1}_r \otimes \mat{I}_K \) \( \mat{I}_K + r \mat{q} (\mat{Q} - \mat{q})^{-1}\)^{-1} \( \vec{1}_r^\intercal \otimes \mat{q} \) \(\mat{I}_r \otimes (\mat{Q} - \mat{q})^{-1}\) \nonumber \\
			&=  \mat{I}_r \otimes (\mat{Q} - \mat{q})^{-1} - \mat{J}_r \otimes \((\mat{Q} + (r-1)\mat{q})^{-1} \mat{q} (\mat{Q} - \mat{q})^{-1}\)
			\end{align}
			
		
		\paragraph{Determinant of $\mat{Q}^{(\rs)}$}
			Therefore we obtain:
			\begin{align}
				\begin{aligned}
				\det{\bSigma} &= \det{\mat{Q}^{(\rs)}}=  \det{\bSigma_r} \det{\mat{S}}\\
					&=\det{\mat{Q} - \mat{q}}^{r-1} \det{\bQ + (r-1)\mat{q}} \det{\mat{Q}^\star - r \mat{m} (\mat{Q} + (r-1)\mat{q})^{-1} \mat{m}^\intercal}
				\end{aligned}
				\label{det_rs}
			\end{align}
		
		\paragraph{Inverse of $\mat{Q}^{(\rs)}$}
			The inverse of the block matrix reads:
			\begin{align*}
				\bSigma^{-1} &= \begin{pmatrix}
					\mat{S}^{-1} & - \mat{S}^{-1} (\vec{1}_r^\intercal
					\otimes \mat{m})  \bSigma_r^{-1} \\
					- \bSigma_r^{-1} (\vec{1}_r
					\otimes \mat{m}^\intercal)   \mat{S}^{-1} & \bSigma_{r}^{-1} + \bSigma_{r}^{-1} (\vec{1}_r
					\otimes \mat{m}^\intercal) \mat{S}^{-1} (\vec{1}_r^\intercal
					\otimes \mat{m}) \bSigma_{r}^{-1}
					\end{pmatrix} 
			\end{align*}
			
			where each block can be computed separately. The top diagonal simply reads
			\begin{align*}
				\mat{S} ^{-1} &= \(\mat{Q}^\star - r \mat{m} (\mat{Q} + (r-1)\mat{q})^{-1} \mat{m}^\intercal\)^{-1}
			\end{align*}		
			and the off-diagonal blocks read:
			\begin{align}
			& - \bSigma_r^{-1} (\vec{1}_r
			\otimes \mat{m}^\intercal)  \mat{S}^{-1} =\\
			&\quad - \vec{1} \otimes \(\mat{Q} + (r-1)\mat{q})^{-1} \mat{m}^\intercal\(\mat{Q}^\star - r \mat{m} (\mat{Q} + (r-1)\mat{q})^{-1} \mat{m}^\intercal\)^{-1} \) \nonumber \\
			& - \mat{S}^{-1} (\vec{1}_r^\intercal
			\otimes \mat{m})  \bSigma_r^{-1} =\\
			& \quad  - \vec{1}_r^\intercal
			\otimes \(\(\mat{Q}^\star - r \mat{m} (\mat{Q} + (r-1)\mat{q})^{-1} \mat{m}^\intercal\)^{-1}  \mat{m} (\mat{Q} + (r-1)\mat{q})^{-1} \)]  \nonumber
			\end{align}
			
			The bottom diagonal block is a bit more involved : 
			\begin{align}
				\bSigma_{r}^{-1} &+ \bSigma_{r}^{-1} (\vec{1}_r
				\otimes \mat{m}^\intercal) \mat{S}^{-1} ( \vec{1}_r^\intercal
				\otimes \mat{m}) \bSigma_{r}^{-1} \nonumber \\
				&= \mat{I}_r \otimes (\mat{Q} - \mat{q})^{-1} - \mat{J}_r \otimes \mat{A}
			\end{align}
			with 
			\begin{align}
				\mat{A} &\equiv (\mat{Q} + (r-1)\mat{q})^{-1} \mat{q} (\mat{Q} - \mat{q})^{-1} \\
				& - (\mat{Q} + (r-1)\mat{q})^{-1} \mat{m}^\intercal\(\mat{Q}^\star - r \mat{m} (\mat{Q} + (r-1)\mat{q})^{-1} \mat{m}^\intercal\)^{-1}\mat{m} (\mat{Q} + (r-1)\mat{q})^{-1}	\nonumber
			\end{align}

			Finally we observe that $\bSigma^{-1}$ is also a block matrix of the form 
			\begin{equation}
				   \bSigma^{-1} = \(\mat{Q}^{(\rs)}\)^{-1}=\begin{bmatrix}
				   \td{\mat{Q}}^\star & \td{\mat{m}} & \cdots & \td{\mat{m}}  \\
				   \td{\mat{m}}^\intercal & \td{\mat{Q}} & \td{\mat{q}} & \cdots \\
				   \vdots & \td{\mat{q}} & \ddots & \td{\mat{q}} \\
				   \td{\mat{m}}^\intercal & \cdots & \td{\mat{q}} & \td{\mat{Q}} \\
				  \end{bmatrix}
			\end{equation}
			with 
			\begin{align*}
				&\td{\mat{Q}}^\star = \(\mat{Q}^\star - r \mat{m} (\mat{Q} + (r-1) \mat{q})^{-1} \mat{m}^\intercal \)^{-1}   \spacecase
				&\td{\mat{m}} = -\(\mat{Q}^\star - r \mat{m} (\mat{Q} + (r-1) \mat{q})^{-1} \mat{m}^\intercal \)^{-1} \mat{m} ( \mat{Q} + (r-1) \mat{q})^{-1}  \spacecase
				&\td{\mat{Q}} = (\mat{Q}-\mat{q})^{-1} - (\mat{Q} +(r-1) \mat{q})^{-1} \mat{q} (\mat{Q}-\mat{q})^{-1} \\
				& \qquad  + ( \mat{Q} + (r-1) \mat{q} )^{-1} \mat{m}^\intercal \\ 
				& \qquad \qquad \times \(\mat{Q}^\star - r \mat{m} ( \mat{Q} + (r-1) \mat{q})^{-1} \mat{m}^\intercal \)^{-1} \mat{m} ( \mat{Q} + (r-1)\mat{q})^{-1}\spacecase
				&\td{\mat{q}} = \td{\mat{Q}} - (\mat{Q}-\mat{q})^{-1}
			\end{align*}
									
			\paragraph{Channel integral $\Psi_{\out}^{(r)}$}
				The quadratic form in $\rp_{\z^a}(\mat{z}^a | \mat{Q}^{(\rs)}$ reads				
				\begin{align*}
					&- \frac{1}{2} \sum_{a,b} \sum_{k,k'} z^{a}_{k} z^b_{k'}(\mat{Q}^{-1})_{k k'}^{ab} \\
					&= -\frac{1}{2} \vec{z}^{\star \intercal} \td{\mat{Q}}^\star \vec{z} 
					-\sum\limits_{a=1}^{r} \vec{z}^{\star \intercal} \td{\mat{m}} \vec{z}^{a}\\
					&
					\qquad \qquad  -\frac{1}{2}\sum\limits_{a=1}^{r} \vec{z}^{a \intercal} \( \td{\mat{Q}} - \td{\mat{q}} \) \vec{z}^{a}
					-\frac{1}{2} \(\sum\limits_{a}^{r} \vec{z}^{a}\)^\intercal \td{\mat{q}} \(\sum\limits_{a}^{r} \vec{z}^{a}\) \,,
				\end{align*}
				and using another Gaussian transformation, see Prop.~\ref{prop:hubbard}, we finally obtain
				\begin{align}
				&\left. \Psi_{\out}^{(r)}(\mat{Q}) \right|_{\rs} =  \displaystyle \int \d y \int_{\bbR^{(r+1) \times K}}  \d \mat{Z} ~ 
			      \rp_{\out}\(y | \mat{Z}\) \rp\(\mat{Z} | \mat{Q}\) \nonumber \\
			      &= \displaystyle \int \d y \int_{\bbR^{(r+1) \times K}}  \d \mat{Z} 
			      \rp_{\out}\(y | \mat{Z}\) e^{- \frac{1}{2} \sum_{a,b=0}^r \sum_{k,k'=1}^K z^{a}_{k} z^b_{k'}(\mat{Q}^{-1})_{k k'}^{ab} } / \( \det{2\pi \mat{Q}}^{(\rs)} \)^{\frac{1}{2}}
				\nonumber \\
				&= \int \d y ~ \EE_{\bxi} \e^{- \frac{1}{2}\log(\det{2\pi \mat{Q}^{(\rs)}} )} \times \int \d \vec{z}^\star ~ \rp_{\out^\star}\(y | \vec{z}^\star \) e^{ -\frac{1}{2} \vec{z}^{\star \intercal} \td{\mat{Q}}^\star \vec{z}^\star} \label{Psi_out_rs} \\ 
				& \times\[ \int \d \vec{z} ~ \rp_{\out}\(y | \vec{z}\) \exp\( - \vec{z}^{\star \intercal} \td{\mat{m}} \vec{z} - \frac{1}{2} \vec{z}^\intercal \( \td{\mat{Q}} - \td{\mat{q}} \) \vec{z}  + \vec{z}^\intercal (-\td{\mat{q}})^{1/2} \bxi\) \]^r\,, \nonumber
				\end{align}
				with $\det{\mat{Q}^{(\rs)}}$ given by \eqref{det_rs}.
			
		\subsubsection{Consistency conditions $r\to 0$: $\Theta(1)$ terms}
			It remains to take the limit $r \to 0^+$ of the expressions for $\Psi_{\w}^{(r)}$ and $\Psi_{\out}^{(r)}$ that are now analytical in $r$.
			%
			First, our assumptions must be consistent and thus we need to check the consistency conditions in the limit $r \to 0$. Indeed, if $\Phi^{(r)}$ is finite we could obtain divergence taking the limit $\lim_{r \to 0} \frac{1}{r} \Phi^{(r)} = \infty$. Therefore to avoid such divergence, we must at least impose that $\lim_{r \to 0} \Phi^{(r)} = 0$:			  
			\begin{align*}
				\lim_{r\to 0} \Phi^{(r)} (\mat{Q},\hat{\mat{Q}}) &= - \tr{\mat{Q}^\star \hat{\mat{Q}}^\star} +\log \Psi_{\w}^{0} (\hat{\mat{Q}}^\star)+\alpha \log \Psi_{\out}^{0}(\mat{Q}^\star)
			\end{align*}	
			with 
			\begin{align*}
				\Psi_{\w}^{0} (\hat{\mat{Q}}^\star) &\equiv \EE_{\vec{w}^\star} \exp\( \vec{w}^{\star \intercal}  \hat{\mat{Q}}^\star \vec{w}^{\star}\)\,, \spacecase 
				\Psi_{\out}^{0}(\mat{Q}^\star) &\equiv \int_\bbR \d y ~  \int \d \vec{z}^\star ~ \rp_{\out^\star}\(y | \vec{z}^\star \) \mN_{\vec{z}^\star}\(\vec{0}, \mat{Q}^\star \)  = 1 \,.
			\end{align*}		
			Taking the saddle point equations over $\mat{Q}^\star$ and $\hat{\mat{Q}}^\star$, imposing the consistency condition $\lim_{r\to 0} \Phi^{(r)} (\mat{Q},\hat{\mat{Q}}) = 0$, we finally obtain
			\begin{align}
				\hat{\mat{Q}}^\star = \mat{0} \andcase \mat{Q}^\star = \EE_{\vec{w}^\star}\[ \vec{w}^{\star \intercal} \vec{w}^{\star} \]\,.
				\label{appendix:replicas:committee:consistency}
			\end{align}
			
		\subsubsection{Replica trick $r \to 0$ limit: $\Theta(r)$ terms}
			Imposing the conditions \eqref{appendix:replicas:committee:consistency} avoids divergence in the replica trick, and we can therefore proceed with the $\Theta(r)$ terms.  
				
		\paragraph{Prior integral $\Psi_{\w}^{(r)}$}
			The limit $r\to 0$ and the derivative of the logarithm of the prior integral \eqref{appendix:replicas:committee:Psi_w_rs} can be trivially computed
			\begin{align*}
			&\lim_{r \to 0} \partial_r \left. \log \Psi_{\w}^{(r)} (\hat{\mat{Q}})\right|_{\rs} \\
			&= \EE_{\bxi, \vec{w}^\star} \\
			& \times \log \[ \EE_{\vec{w}}  \exp \(\[  \vec{w}^{\star \intercal} \hat{\mat{m}} \vec{w} - \frac{1}{2} \vec{w}^\intercal (\hat{\mat{Q}} + \hat{\mat{q}}) \vec{w} + \bxi^\intercal \hat{\mat{q}}^{1/2}  \vec{w} \] \) \]\,,
			\end{align*}	
			with $\bxi \sim \mN(\vec{0},\mat{I}_K)$ and $\vec{w}^\star \sim \rP_{\w^\star}$. 
			To conclude, we can symmetrize and decouple the \emph{teacher} and \emph{student} expectations $\EE_{\vec{w}^\star},\EE_{\vec{w}}$. By performing the change of variable $\bxi \leftarrow \bxi + \hat{\mat{q}}^{-1/2} \hat{\mat{m}} \vec{w}^\star$. We finally obtain		
			\begin{align}
			&\lim_{r \to 0} \partial_r \left. \log \Psi_{\w}^{(r)} (\hat{\mat{Q}})\right|_{\rs} \nonumber \\ 
			&=  \EE_{\bxi, \vec{w}^\star}\exp\(-\frac{1}{2} \vec{w}^{\star \intercal} \hat{\mat{m}}^\intercal \hat{\mat{q}}^{-1}\hat{\mat{m}} \vec{w}^\star + \bxi^\intercal \hat{\mat{q}}^{-1/2} \hat{\mat{m}} \vec{w}^\star  \)  \label{appendix:replicas:committee:log_Psi_w} \\
			& \hspace{1cm} \times \log \[ \EE_{\vec{w}}  \exp \(\[ - \frac{1}{2}\vec{w}^{\intercal} (\hat{\mat{Q}} + \hat{\mat{q}})\vec{w}  + \bxi^\intercal \hat{\mat{q}}^{1/2} \vec{w} \] \) \] \nonumber \\
			&\equiv \EE_{\bxi, \vec{w}^\star} \mZ_{\w^\star}\(\hat{\mat{m}}\hat{\mat{q}}^{-1/2} \bxi, \hat{\mat{m}}^\intercal \hat{\mat{q}}^{-1}\hat{\mat{m}} \) \log \mZ_\w\(\hat{\mat{q}}^{1/2} \bxi , \hat{\mat{Q}} + \hat{\mat{q}}\)\,, \nonumber
			\end{align}
			with the corresponding denoising distribution $\rQ_\w$ and functions $\mZ_{\w^\star},\mZ_{\w}$ defined by
		\begin{align}
			\begin{aligned}
			\rQ_{\w}(\vec{w}; \bgamma, \bLambda) & \equiv \displaystyle \frac{\rP_{\w}(\vec{w}) }{\mZ_{\w} (\bgamma,\bLambda)} e^{ - \frac{1}{2} \vec{w}^\intercal \bLambda \vec{w}  + \bgamma^\intercal \vec{w}  }\,, \\
			\mZ_{\w} (\bgamma,\bLambda) &\equiv \EE_{\vec{w}\sim \rP_\w} \[ e^{ - \frac{1}{2} \vec{w}^\intercal \bLambda \vec{w}  + \bgamma^\intercal \vec{w}  } \] = \int_{\bbR^K} \d \vec{w} ~ \rp_{\w}(\vec{w})  e^{ - \frac{1}{2} \vec{w}^\intercal \bLambda \vec{w}  + \bgamma^\intercal \vec{w}  }\,,
			\end{aligned}
		\label{distributions:committee_prior}
		\end{align}
		respectively with distribution $\rP_{\w^\star}$ and $\rP_\w$.

		\paragraph{Prior integral $\Psi_{\out}^{(r)}$}
			The limit $r\to 0$ and the derivative of the logarithm of the channel integral \eqref{Psi_out_rs} is more tricky. First, the limit of the determinant \eqref{det_rs} simplifies easily and yields
			\begin{align*}
				\det{\mat{Q}^{(\rs)}} \underlim{r}{0} \det{\mat{Q}^\star}
			\end{align*}
			and the matrix elements of $\(\mat{Q}^{(\rs)}\)^{-1}$ in this limit become
			\begin{align*}
				\td{\mat{Q}}^\star & \underlim{r}{0} \(\mat{Q}^\star\)^{-1}\,,  \spacecase
				\td{\mat{m}} & \underlim{r}{0} -\(\mat{Q}^\star \)^{-1} \mat{m} ( \mat{Q} - \mat{q})^{-1}\,,  \spacecase
				\td{\mat{q}} & \underlim{r}{0} - (\mat{Q} - \mat{q})^{-1}\(\mat{q} -  \mat{m}^\intercal \(\mat{Q}^\star \)^{-1} \mat{m} \)( \mat{Q}-\mat{q})^{-1}\,, \spacecase
				\td{\mat{Q}} & \underlim{r}{0}  \td{\mat{q}} + (\mat{Q}-\mat{q})^{-1}\,.
			\end{align*}
			
			To take properly the $r\to 0$ limit, let us first perform the change of variable $\bxi \leftarrow \bxi - (-\td{\mat{q}})^{1/2} \td{\mat{m}}^\intercal \vec{z}^\star $ in \eqref{Psi_out_rs} so that 
			\begin{align}
			\begin{aligned}
			\left. \Psi_{\out}^{(r)}(\mat{Q}) \right|_{\rs} &= \displaystyle  \int_\bbR \d y \EE_{\bxi} J_0 (y, \bxi, r) \times J_1(y, \bxi, r)^r \\
			J_0 (y, \bxi, r) &=  \e^{- \frac{1}{2}\log(\det{2\pi \mat{Q}^{(\rs)}} )} \\
			& \times \int_{\bbR^K} \d \vec{z}^\star \rp_{\out^\star} ~ \(y | \vec{z}^\star \) e^{ -\frac{1}{2} \vec{z}^{\star \intercal} \(\td{\mat{Q}}^\star - \td{\mat{m}}^\intercal \td{\mat{q}}^{-1} \td{\mat{m}} \) \vec{z}^\star + \bxi^\intercal (-\td{\mat{q}})^{1/2} \td{\mat{m}}^\intercal \vec{z}^{\star}   } \\
			J_1(y, \bxi, r) &= \int_{\bbR^K} \d \vec{z}  ~ \rp_{\out}\(y | \vec{z}\) e^{-\frac{1}{2}  \vec{z}^\intercal \( \td{\mat{Q}} - \td{\mat{q}} \) \vec{z}  + \vec{z}^\intercal (-\td{\mat{q}})^{1/2} \bxi }
			\end{aligned}
			\end{align}	
			so that the integral may be written as follows:
			\begin{align*}
				&\log\(\Psi_{\out}^{(r)}\) = \log \int \d y \EE_{\bxi} J_0(y,\bxi, r) J_1(y,\bxi, r)^r\\
				&=  \log \int_\bbR \d y  \EE_{\bxi} \[ J_0(y, \bxi, 0) + r \partial_r J_0(y, \bxi, 0) + r J_0(y, \bxi, 0) \log J_1(y, \bxi, 0) + \Theta(r^2) \]\\
				&= \log \(1 + r \int_\bbR \d y \times \EE_{\bxi} \[ \partial_r J_0(y, \bxi, 0) + J_0(y, \bxi, 0) \log J_1(y, \bxi, 0) + \Theta(r^2)  \]  \)\\
				&=  r \int_\bbR \d y \times \EE_{\bxi} \[ \partial_r J_0(y, \bxi, 0) + J_0(y, \bxi, 0) \log J_1(y, \bxi, 0)  \] + \Theta(r^2) 
			\end{align*}

	where we used the consistency condition $\int_\bbR \d y \times \EE_{\bxi} J_0(y, \bxi, 0)=1$. The first term can be evaluated similarly
		\begin{align*}
			&\int_\bbR \d y \times \EE_{\bxi} \partial_r J_0(y, \bxi, 0) = \partial_r \[ \int_\bbR \d y \times \EE_{\bxi} J_0(y, \bxi, r) \]_{r=0} \\
			&=  \partial_r \[  \e^{- \frac{1}{2}\log(\det{2\pi \mat{Q}^{(\rs)}} )} \int_{\bbR^K} \d \vec{z}^\star e^{ -\frac{1}{2} \vec{z}^{\star \intercal} \(\td{\mat{Q}}^\star \) \vec{z}^\star}  \]_{r=0} \\
			&= \partial_r \[  \e^{- \frac{1}{2}\log(\det{ \td{\mat{Q}}^\star}  \det{\mat{Q}^{(\rs)}} )} \]_{r=0} = ... = 0 	
			\end{align*}

		Finally we obtain:
			\begin{align*}
			    &\lim\limits_{r\to 0} \partial_r \log\(\Psi_{\out}^{(r)}\) = \int_\bbR \d y \EE_{\bxi}  J_0 (y,\bxi) \log J_1 (y,\bxi) = \lim_{r\to 0} \int_\bbR \d y \EE_{\bxi}  J_0 (y,\bxi) J_1 (y,\bxi)^r
			\end{align*}
			with
			\begin{align*}
			    &J_{0}(y,\bxi) \equiv \e^{- \frac{1}{2}\log(\det{2\pi \mat{Q}^\star} )} \int_{\bbR^K} \d \vec{z}^\star \rp_{\out^\star}\left(y|\vec{z}^\star \right) e^{-\frac{1}{2} \vec{z}^{\star \intercal} \mat{N} \vec{z}^{\star} + \bxi^\intercal \mat{M} \vec{z}^{\star}   }\\
			    &J_{1}(y,\bxi) \equiv \int_{\bbR^K} \d \vec{z} ~\rp_{\out}\(y|\vec{z}\)e^{-\frac{1}{2} \vec{z}^\intercal \(\mat{Q}- \mat{q}\)^{-1} \vec{z} + \vec{z}^\intercal \mat{P} \bxi } \\
			    &= \EE_{\vec{z}\sim\mN\(\vec{0}, \mat{I}_K\)} ~\rp_{\out}\(y|\(\mat{Q}-\mat{q}\)^{1/2} \vec{z}\)e^{\bxi^\intercal \mat{P} \(\mat{Q}-\mat{q}\)^{1/2} \vec{z} }
			\end{align*}
			where we changed the variable $\vec{z} \leftarrow \(\mat{Q}-\mat{q}\)^{-1/2} \vec{z}$ and used the fact that with the second formulation the determinant term at the power $r$ goes away in the limit $r\to 0$ and
			with 
			\begin{align*}
				\mat{P} &\equiv (-\td{\mat{q}})^{1/2} = (\mat{Q}-\mat{q})^{-1/2}\( \mat{q} - \mat{m}^\intercal (\mat{Q}^\star)^{-1} \mat{m} \)^{1/2} (\mat{Q}-\mat{q})^{-1/2}\\
				\mat{M} &\equiv  (-\td{\mat{q}})^{-1/2} \td{\mat{m}}^\intercal = - (-\td{\mat{q}})^{-1/2} ( \mat{Q} - \mat{q})^{-1} \mat{m}^\intercal \(\mat{Q}^\star \)^{-1}  \\
				&= -\mat{P}^{-1} ( \mat{Q} - \mat{q})^{-1} \mat{m}^\intercal \(\mat{Q}^\star \)^{-1}\\
				\mat{N} &\equiv \td{\mat{Q}}^\star - \td{\mat{m}}^\intercal \td{\mat{q}}^{-1} \td{\mat{m}}   \\
						&= (\mat{Q}^\star)^{-1} + (\mat{Q}^\star)^{-1} \mat{m} (\mat{Q}-\mat{q})^{-1} (\mat{Q}-\mat{q}) \( \mat{q} - \mat{m}^\intercal (\mat{Q}^\star)^{-1} \mat{m} \)^{-1} (\mat{Q}-\mat{q}) (\mat{Q}-\mat{q})^{-1} \mat{m}^\intercal (\mat{Q}^\star)^{-1} \\
						&= (\mat{Q}^\star)^{-1}\[ \mat{I}_K  + \mat{m} \( \mat{q} - \mat{m}^\intercal (\mat{Q}^\star)^{-1} \mat{m} \)^{-1} \mat{m}^\intercal (\mat{Q}^\star)^{-1} \] \\
						&=  (\mat{Q}^\star)^{-1}\[ \mat{I}_K  +  \[ (\mat{Q}^\star)^{-1} \mat{m}^{-\intercal} \mat{q}  \mat{m}^{-1}  - \mat{I}_K\]^{-1}\] \\
						&= \[\mat{Q}^\star - \mat{m} \mat{q}^{-1} \mat{m}^\intercal \]^{-1}
			\end{align*}

			Then we rescale $\vec{z}^\star \leftarrow  \mat{N}^{1/2} \vec{z}^{\star}$ such that 
			\begin{align*}
				J_{0}(y,\bxi) &= \e^{- \frac{1}{2}\log(\det{2\pi \mat{Q}^\star} )} \int_{\bbR^K} \d \vec{z}^\star \rp_{\out^\star}\left(y|\vec{z}^\star \right) e^{-\frac{1}{2} \vec{z}^{\star \intercal} \mat{N} \vec{z}^{\star} + \bxi^\intercal \mat{M} \vec{z}^{\star}   }\\
				&=  \e^{- \frac{1}{2}\log(\det{2\pi \mat{Q}^\star} )} \e^{\frac{1}{2}\log\(\det{\mat{Q}^\star - \mat{m} \mat{q}^{-1} \mat{m}^\intercal}\)} \\
				& \times\EE_{\vec{z}^\star \sim\mN\(\vec{0}, \mat{I}_K\)} \[  \rp_{\out^\star}\left(y| \(\mat{Q}^\star - \mat{m} \mat{q}^{-1} \mat{m}^\intercal\)^{1/2} \vec{z}^\star \right) e^{\bxi^\intercal \mat{M} \mat{N}^{-1/2} \vec{z}^{\star}   } \]
			\end{align*}
			
			Moreover notice that:
			\begin{align*}
				\mat{M} \mat{N}^{-1/2} &= \mat{P}^{-1} ( \mat{Q} - \mat{q})^{-1} \mat{m}^\intercal \(\mat{Q}^\star \)^{-1} \[\mat{Q}^\star - \mat{m} \mat{q}^{-1} \mat{m}^\intercal \]^{1/2}\\
				&= \mat{P} \mat{P}^{-2} ( \mat{Q} - \mat{q})^{-1} \mat{m}^\intercal \(\mat{Q}^\star \)^{-1} \[\mat{Q}^\star - \mat{m} \mat{q}^{-1} \mat{m}^\intercal \]^{1/2}\\
				&= \mat{P} (\mat{Q}-\mat{q})\( \mat{q} - \mat{m}^\intercal (\mat{Q}^\star)^{-1} \mat{m} \)^{-1} (\mat{Q}-\mat{q}) ( \mat{Q} - \mat{q})^{-1} \mat{m}^\intercal \(\mat{Q}^\star \)^{-1} \[\mat{Q}^\star - \mat{m} \mat{q}^{-1} \mat{m}^\intercal \]^{1/2}\\
				&= \mat{P} (\mat{Q}-\mat{q})^{1/2} \mat{R}\\
			\end{align*}
			with 
			\begin{align*}
				\mat{R} &= (\mat{Q}-\mat{q})^{1/2} \mat{q}^{-1} \mat{m}^\intercal\[\mat{Q}^\star - \mat{m} \mat{q}^{-1} \mat{m}^\intercal \]^{-1/2}
			\end{align*}
			and notice that $\mat{R} = \mat{I}_K$ in the Bayes-optimal setting. Therefore:
			
			\begin{align*}
				J_{0}(y,\bxi)
				&=  \e^{- \frac{1}{2}\log(\det{2\pi \mat{Q}^\star} )} \e^{\frac{1}{2}\log\(\det{\mat{Q}^\star - \mat{m} \mat{q}^{-1} \mat{m}^\intercal}\)} \\
				& \times\EE_{\vec{z}^\star \sim\mN\(\vec{0}, \mat{I}_K\)} \[  \rp_{\out^\star}\left(y| \(\mat{Q}^\star - \mat{m} \mat{q}^{-1} \mat{m}^\intercal\)^{1/2} \vec{z}^\star \right) e^{\bxi^\intercal \mat{P} (\mat{Q}-\mat{q})^{1/2} \mat{R} \vec{z}^{\star}   } \]
			\end{align*}
			
			As a summary, we have:
			\begin{align*}
			    &\lim\limits_{r\to 0} \partial_r \log\(\Psi_{\out}^{(r)}\) = \lim_{r\to 0} \int_\bbR \d y \EE_{\bxi}  J_0 (y,\bxi) J_1 (y,\bxi)^r
			\end{align*}
			with
			\begin{align*}
			    J_{0}(y,\bxi)
				&=  \e^{- \frac{1}{2}\log(\det{2\pi \mat{Q}^\star} )} \e^{\frac{1}{2}\log\(\det{\mat{Q}^\star - \mat{m} \mat{q}^{-1} \mat{m}^\intercal}\)} \\
				& \times\EE_{\vec{z}^\star \sim\mN\(\vec{0}, \mat{I}_K\)} \[  \rp_{\out^\star}\left(y| \(\mat{Q}^\star - \mat{m} \mat{q}^{-1} \mat{m}^\intercal\)^{1/2} \vec{z}^\star \right) e^{\bxi^\intercal \mat{P} (\mat{Q}-\mat{q})^{1/2} \mat{R} \vec{z}^{\star}   } \]\\
 				J_{1}(y,\bxi) &= \EE_{\vec{z}\sim\mN\(\vec{0}, \mat{I}_K\)} ~\rp_{\out}\(y|\(\mat{Q}-\mat{q}\)^{1/2} \vec{z}\)e^{\bxi^\intercal \mat{P} \(\mat{Q}-\mat{q}\)^{1/2} \vec{z} }
			\end{align*}
				
			We can shift $\vec{z} \leftarrow \vec{z} - \(\mat{Q} - \mat{q}\)^{1/2} \mat{P} \bxi$ to obtain 
			\begin{align*}
				J_{1}(y,\bxi) &= e^{\frac{1}{2} \bxi^\intercal \mat{P}\(\mat{Q}-\mat{q}\)^{1/2} \(\mat{Q}-\mat{q}\)^{1/2} \mat{P} \bxi} \\
				& \times \EE_{\vec{z}\sim\mN\(\vec{0}, \mat{I}_K\)} ~\rp_{\out}\(y|\(\mat{Q}-\mat{q}\)^{1/2} \vec{z} + \(\mat{Q}-\mat{q}\) \mat{P} \bxi \)
			\end{align*}
			and $\vec{z}^\star \leftarrow \vec{z}^\star - \(\mat{Q}^\star - \mat{m} \mat{q}^{-1} \mat{m}^\intercal\)^{-1/2} \mat{m} \mat{q}^{-1} \(\mat{Q}-\mat{q}\) \mat{P} \bxi  =  \vec{z}^\star - \mat{R}^\intercal \(\mat{Q}-\mat{q}\)^{1/2}\mat{P} \bxi $:
			\begin{align*}
				J_{0}(y,\bxi)
				&=  \e^{- \frac{1}{2}\log(\det{2\pi \mat{Q}^\star} )} \e^{\frac{1}{2}\log\(\det{\mat{Q}^\star - \mat{m} \mat{q}^{-1} \mat{m}^\intercal}\)} \\
				& \times\EE_{\vec{z}^\star \sim\mN\(\vec{0}, \mat{I}_K\)} \[  \rp_{\out^\star}\left(y| \(\mat{Q}^\star - \mat{m} \mat{q}^{-1} \mat{m}^\intercal\)^{1/2} \vec{z}^\star \right) e^{\bxi^\intercal \mat{P} (\mat{Q}-\mat{q})^{1/2} \mat{R} \vec{z}^{\star}   } \]\\
				&=  \e^{- \frac{1}{2}\log(\det{2\pi \mat{Q}^\star} )} \e^{\frac{1}{2}\log\(\det{\mat{Q}^\star - \mat{m} \mat{q}^{-1} \mat{m}^\intercal}\)} e^{\frac{1}{2}\bxi^\intercal \mat{P} \(\mat{Q}-\mat{q}\)^{1/2} \mat{R} \mat{R}^\intercal \(\mat{Q}-\mat{q}\)^{1/2} \mat{P} \bxi  }\\
				& \times\EE_{\vec{z}^\star \sim\mN\(\vec{0}, \mat{I}_K\)} \[  \rp_{\out^\star}\left(y| \(\mat{Q}^\star - \mat{m} \mat{q}^{-1} \mat{m}^\intercal\)^{1/2} \vec{z}^\star + \mat{m}\mat{q}^{-1} \(\mat{Q}-\mat{q}\) \mat{P} \bxi \right)\]\\
			\end{align*} 
			
			We obtain (taking the limit $r\to 0$ to the term in front of $J_{1}$): 
			
			\begin{align*}
				& \lim\limits_{r\to 0} \partial_r \log\(\Psi_{\out}^{(r)}\) \\
				&= \int_\bbR \d y \int_{\bbR^K} \d\bxi  \e^{- \frac{1}{2}\log(\det{\mat{Q}^\star} )} \e^{\frac{1}{2}\log\(\det{\mat{Q}^\star - \mat{m} \mat{q}^{-1} \mat{m}^\intercal}\)} e^{-\frac{1}{2}\bxi^\intercal \[ \mat{I}_K - \mat{P} \(\mat{Q}-\mat{q}\)^{1/2} \mat{R} \mat{R}^\intercal \(\mat{Q}-\mat{q}\)^{1/2} \mat{P} \] \bxi  } \\
				& \times \EE_{\vec{z}^\star \sim\mN\(\vec{0}, \mat{I}_K\)} \[  \rp_{\out^\star}\left(y| \(\mat{Q}^\star - \mat{m} \mat{q}^{-1} \mat{m}^\intercal\)^{1/2} \vec{z}^\star + \mat{m}\mat{q}^{-1} \(\mat{Q}-\mat{q}\) \mat{P} \bxi \right)\] \\
				& \times \log \[ \EE_{\vec{z}\sim\mN\(\vec{0}, \mat{I}_K\)} ~\rp_{\out}\(y|\(\mat{Q}-\mat{q}\)^{1/2} \vec{z} + \(\mat{Q}-\mat{q}\) \mat{P} \bxi \) \]
			\end{align*} 
			
			Computing 
			
			\begin{align*}
				& \[ \mat{I}_K - \mat{P} \(\mat{Q}-\mat{q}\)^{1/2} \mat{R} \mat{R}^\intercal \(\mat{Q}-\mat{q}\)^{1/2} \mat{P} \]\\
				&= \mat{P}\[ \mat{P}^{-2} - \(\mat{Q}-\mat{q}\)^{1/2} \mat{R} \mat{R}^\intercal \(\mat{Q}-\mat{q}\)^{1/2}  \]\mat{P} \\
				&= \mat{P} (\mat{Q}-\mat{q}) \[ \( \mat{q} - \mat{m}^\intercal (\mat{Q}^\star)^{-1} \mat{m} \)^{-1} - \mat{q}^{-1} \mat{m}^\intercal\[\mat{Q}^\star - \mat{m} \mat{q}^{-1} \mat{m}^\intercal \]^{-1} \mat{m} \mat{q}^{-1} \] \(\mat{Q}-\mat{q}\)\mat{P}\\
				&= \mat{P} (\mat{Q}-\mat{q}) \mat{q}^{-1} (\mat{Q}-\mat{q}) \mat{P}
			\end{align*}
			
			
			By finally changing the Gaussian measure $\bxi \leftarrow \mat{q}^{-1/2} \(\mat{Q}-\mat{q}\) \mat{P} \bxi $ we obtain
			
			\begin{align*}
				& \lim\limits_{r\to 0} \partial_r \log\(\Psi_{\out}^{(r)}\) \\
				&= \exp\(C\) \times \int_\bbR \d y \EE_{\bxi} \\
				& \times \EE_{\vec{z}^\star \sim\mN\(\vec{0}, \mat{I}_K\)} \[  \rp_{\out^\star}\left(y| \(\mat{Q}^\star - \mat{m} \mat{q}^{-1} \mat{m}^\intercal\)^{1/2} \vec{z}^\star + \mat{m}\mat{q}^{-1/2}\bxi \right)\] \\
				& \times \log \[ \EE_{\vec{z}\sim\mN\(\vec{0}, \mat{I}_K\)} ~\rp_{\out}\(y|\(\mat{Q}-\mat{q}\)^{1/2} \vec{z} + \mat{q}^{1/2} \bxi \) \]
			\end{align*} 
			
			with 
			\begin{align*}
				C &= - \frac{1}{2}\log(\det{\mat{Q}^\star} ) + \frac{1}{2}\log\(\det{\mat{Q}^\star - \mat{m} \mat{q}^{-1} \mat{m}^\intercal}\)\\
				& -\frac{1}{2} \log \det{\mat{P}^2} - \log \det{\mat{Q}-\mat{q}} +\frac{1}{2} \log \det{\mat{q}} \\
				&= - \frac{1}{2}\log(\det{\mat{Q}^\star} ) + \frac{1}{2}\log\(\det{\mat{Q}^\star - \mat{m} \mat{q}^{-1} \mat{m}^\intercal}\)\\
				& -\frac{1}{2} \log \det{(\mat{Q}-\mat{q})\( \mat{q} - \mat{m}^\intercal (\mat{Q}^\star)^{-1} \mat{m} \) (\mat{Q}-\mat{q})} - \log \det{\mat{Q}-\mat{q}} +\frac{1}{2} \log \det{\mat{q}} \\
				&= - \frac{1}{2}\log(\det{\mat{Q}^\star} ) + \frac{1}{2}\log\(\det{\mat{Q}^\star - \mat{m} \mat{q}^{-1} \mat{m}^\intercal}\)\\
				& -\frac{1}{2} \log \det{\( \mat{q} - \mat{m}^\intercal (\mat{Q}^\star)^{-1} \mat{m} \)} +\frac{1}{2} \log \det{\mat{q}} \\
				&= - \frac{1}{2}\log(\det{\mat{Q}^\star} ) - \frac{1}{2} \log \det{\mat{m}\mat{m}^{-1}} - \frac{1}{2} \log \det{\mat{q}} + \frac{1}{2}\log(\det{\mat{Q}^\star} ) + \frac{1}{2}\log(\det{\mat{q}} )\\
				&= 0			 
			\end{align*}
			
			We obtain: 
			\begin{align}
			\begin{aligned}
				& \lim_{r\to 0} \partial_r \left. \log  \Psi_{\out}^{(r)} (\mat{Q})\right|_{\rs}\\
				&=  \int_\bbR \d y  \EE_{\bxi} \\
				& \times \EE_{\vec{z}^\star \sim\mN\(\vec{0}, \mat{I}_K\)} \[  \rp_{\out^\star}\left(y| \(\mat{Q}^\star - \mat{m} \mat{q}^{-1} \mat{m}^\intercal\)^{1/2} \vec{z}^\star + \mat{m}\mat{q}^{-1/2}\bxi \right)\] \\
				& \times \log \[ \EE_{\vec{z}\sim\mN\(\vec{0}, \mat{I}_K\)} ~\rp_{\out}\(y|\(\mat{Q}-\mat{q}\)^{1/2} \vec{z} + \mat{q}^{1/2} \bxi \) \] \\
				&= \int_\bbR \d y ~ \EE_{\bxi} \mZ_{\out^\star}\( \mat{m}\mat{q}^{-1/2}\bxi, \mat{Q}^\star - \mat{m} \mat{q}^{-1} \mat{m}^\intercal \)  \log \mZ_{\out}\(\mat{q}^{1/2} \bxi, \mat{Q}-\mat{q}\)\,,
			\end{aligned}
			\end{align}

			where the denoising distribution $\rQ_{\out}$ and functions $\mZ_{\out^\star},\mZ_{\out}$ are defined by
			\begin{align}
			\begin{aligned}
				\rQ_{\out} (\vec{z}; y, \bomega, \bV ) & \equiv  \displaystyle\frac{\rP_{\out}\( y | \vec{z}\) }{\mZ_{\out}(y, \bomega, \bV)}  \frac{e^{ -\frac{1}{2}  \(\vec{z} - \bomega\)^\intercal \bV^{-1} \(\vec{z} - \bomega\)  }}{\sqrt{\det{2\pi \bV}}}\,, \\
			\mZ_{\out}(y, \bomega, \bV) &\equiv \EE_{\vec{z} \sim \mN(\vec{0},\rI_K)}\[ \rP_{\out}\(y | \bV^{1/2} \vec{z} + \bomega \) \] \\ 
			&= \int_{\bbR^K} \d \vec{z} ~ \rp_{\out}\(y | \vec{z} \) \frac{e^{ -\frac{1}{2}  \(\vec{z} - \bomega\)^\intercal \bV^{-1} \(\vec{z} - \bomega\)  }}{\sqrt{\det{2\pi \bV}}}\,.
			\end{aligned}
			\label{distributions:committee_channel}
			\end{align}
			 for the distributions $\rP_{\out^\star}$, $\rP_{\out}$.


\subsection{Summary}

	\subsubsection{Summary - Mismatched case}
		In the mismatched case, where the teacher and the student have not the same prior distributions, we finally obtain the replica symmetric free entropy $\Phi_{\rs}$ for the committee machine hypothesis class:
		\begin{align}
			&\Phi_{\rs}(\alpha) \equiv \EE_{\vec{y},\mat{X}} \[ \lim_{\ndim \to \infty} \frac{1}{\ndim} \log\( \mZ_\ndim\(\vec{y}, \mat{X}\) \) \] \label{appendix:replicas:free_entropy_non_bayes} \nonumber \\
			&=  \extr_{\mat{Q},\hat{\mat{Q}}, \mat{q}, \hat{\mat{q}}, \mat{m}, \hat{\mat{m}}} \left \{  - \tr{ \mat{m} \hat{\mat{m}}} + \frac{1}{2} \tr{ \mat{Q} \hat{\mat{Q}}} + \frac{1}{2} \tr{  \mat{q} \hat{\mat{q}}}  \right. \\
			& \left. \qquad \qquad \qquad \qquad  \qquad \qquad + \Psi_\w\(\hat{\mat{Q}},\hat{\mat{m}},\hat{\mat{q}}  \) + \alpha \Psi_\out\( \mat{Q}, \mat{m}, \mat{q};  \brho_{\w^\star}\) \right\} \nonumber \,,
		\end{align}
		where $\brho_{\w^\star} \equiv \lim_{\ndim \to \infty} \mat{Q}^\star = \lim_{\ndim \to \infty} \EE_{\vec{w}^\star} \frac{1}{\ndim} \mat{W}^{\star\intercal} \mat{W}^{\star} $ and the channel and prior integrals are defined by
		\begin{align}
			\Psi_\w\(\hat{\mat{Q}},\hat{\mat{m}},\hat{\mat{q}}  \) &\equiv \EE_{\bxi} \[ \mZ_{\w^\star}\( \hat{\mat{m}}  \hat{\mat{q}}^{-1/2}  \bxi , \hat{\mat{m}} \hat{\mat{q}}^{-1} \hat{\mat{m}}  \) \right. \label{appendix:replicas:committee:free_entropy_terms_non_bayes} \\
			& \left. \qquad \qquad \qquad \qquad \times  \log \mZ_{\w} \(  \hat{\mat{q}}^{1/2} \bxi, \hat{\mat{Q}} + \hat{\mat{q}} \)   \]\,, \nonumber \spacecase
			\Psi_\out\(\mat{Q},\mat{m}, \mat{q}; \brho_{\w^\star}\) &\equiv \EE_{y, \bxi } \[ \mZ_{\out^\star} \( y,  \mat{m} \mat{q}^{-1/2} \bxi, \brho_{\w^\star} - \mat{m} \mat{q}^{-1} \mat{m}^\intercal  \)\right. \nonumber \\
			& \left. \qquad \qquad \qquad \qquad \times \log \mZ_{\out} \( y,  \mat{q}^{1/2} \bxi, \mat{Q} - \mat{q} \)  \]\,,
		\end{align}
		where again $\mZ_{\out^\star},\mZ_{\w^\star}$ and $\mZ_{\out},\mZ_{\w}$ are defined in \eqref{distributions:committee_prior}- \eqref{distributions:committee_channel} and depend respectively on channel and prior distributions of the \emph{teacher} and \emph{student}.


	\subsubsection{Bayes optimal MMSE estimation}
		For MMSE estimation in the Bayes-optimal setting, the student has access to the ground truth distributions of the teacher $\rP_\out\(\vec{y} | \mat{Z} \) = \rP_{\out^\star}\(\vec{y} | \mat{Z}\) $ and $\rP_\w\(\mat{W}\) = \rP_{\w^\star}(\mat{W})$, and therefore $\mZ_{\out}=\mZ_{\out^\star}$, $\mZ_{\w}=\mZ_{\w^\star}$. 
		In this idealistic setting, the Nishimori conditions imply that 
		\begin{align}
		\begin{aligned}
			\mat{Q} &= \mat{Q}_{\w^\star}\,, && \hat{\mat{Q}}=\mat{0}\,, && \mat{m}= \mat{q} \equiv \mat{q}_\bayes \,, && \hat{\mat{m}}=\hat{\mat{q}} \equiv  \hat{\mat{q}}_\bayes \,.
		\end{aligned}
			\label{appendix:replicas:committee:nishimori}
		\end{align}
		Therefore the free entropy in \eq\eqref{appendix:replicas:committee:free_entropy_terms_non_bayes} simplifies as an optimization problem over the \emph{overlaps} $\mat{q}_\bayes,\hat{\mat{q}}_\bayes \in \bbR^{K\times K}$
		\begin{align}
			\Phi_{\rs}^\bayes (\alpha) &=  \extr_{\mat{q}_\bayes,\hat{\mat{q}}_\bayes} \left \{ - \frac{1}{2} \tr{\mat{q}_\bayes \hat{\mat{q}}_\bayes}  + \Psi_{\w}^\bayes\(\hat{\mat{q}}_\bayes  \) + \alpha \Psi_{\out}^\bayes\(\mat{q}_\bayes; \brho_{\w^\star}\) \right\} \,,
			\label{appendix:free_entropy_bayes}
		\end{align}
		with free entropy terms $\Psi_{\w}^\bayes$ and $\Psi_{\out}^\bayes$ given by
		\begin{align}
				\Psi_{\w}^\bayes\(\hat{\mat{q}}\) &= \EE_{\xi} \[ \mZ_{\w^\star} \(  \hat{\mat{q}}^{1/2}\bxi,   \hat{\mat{q}} \)  \log \mZ_{\w^\star} \(  \hat{\mat{q}}^{1/2}\bxi,   \hat{\mat{q}} \)   \] \,, \nonumber \spacecase 
				\Psi_{\out}^\bayes \(\mat{q}; \brho_{\w^\star}\) &= \EE_{y, \bxi } \[ \mZ_{\out} \( y,  \mat{q}^{1/2} \bxi, \brho_{\w^\star} - \mat{q} \) \right. \\
				& \left. \qquad \qquad \qquad \qquad  \times  \log \mZ_{\out} \( y,  \mat{q}^{1/2}\bxi, \brho_{\w^\star} - \mat{q} \)  \]\,. \nonumber
		\end{align} 

\paragraph{Application to the GLM}
For the GLM hypothesis class, the same equations are valid if we take $K=1$ for both the teacher and the student. As a result, we recover the replica symmetric free entropy in the Bayes-optimal setting rigorously proven in \cite{barbier2017phase}.

\subsection{Fixed point equations}
\label{appendix:replica_computation:committee:fixed_point}

The overlaps parameters, such as $\mat{m}, \mat{q}$, play a crucial role since they measure the performances of the statistical estimation. Their behaviours are respectively characterized by the extremization of the free entropy \eqref{appendix:replicas:free_entropy_non_bayes} in the mismatched setting and \eqref{appendix:free_entropy_bayes} in the Bayes-optimal case.
In this section, we give the expressions of the corresponding fixed point equations, whose derivations can be found in \cite{aubin2020generalization} \App IV.4-5 for $K=1$ which can be extended to $K \ge 1$.

\subsubsection{Mismatched setting}
\label{appendix:fixed_point_erm}
Extremizing the free entropy eq.~\eqref{appendix:replicas:free_entropy_non_bayes}, we easily obtain the set of six fixed point equations
\begin{align}
\begin{aligned}
	\hat{\mat{Q}} &= -2 \alpha \partial_{\mat{Q}} \Psi_\out\(\mat{Q},\mat{m}, \mat{q}; \brho_{\w^\star}\) \,, \qquad 
			&& \mat{Q} = - 2  \partial_{\hat{\mat{Q}}} \Psi_{\w}\(\hat{\mat{Q}},\hat{\mat{m}},\hat{\mat{q}}  \)\spacecase
	\hat{\mat{q}} &= -2 \alpha \partial_{\mat{q}} \Psi_\out\(\mat{Q},\mat{m}, \mat{q}; \brho_{\w^\star}\)\,, 
			&& \mat{q} =   -2\partial_{\hat{\mat{q}}} \Psi_\w\(\hat{\mat{Q}},\hat{\mat{m}},\hat{\mat{q}}  \) \,,\spacecase
	\hat{\mat{m}} &= \alpha \partial_{\mat{m}} \Psi_\out\(\mat{Q},\mat{m}, \mat{q}; \brho_{\w^\star}\)\,,
			&& \mat{m} = \partial_{\hat{\mat{m}}} \Psi_\w\(\hat{\mat{Q}},\hat{\mat{m}},\hat{\mat{q}}  \) \,.
\end{aligned}
\label{appendix:se_equations_generic_not_simplified}
\end{align}
Interestingly, these equations can be reformulated as functions of $\mZ_{\out^\star}$, $\mZ_{\w^\star}$ and the denoising functions $f_{\out^\star}, f_{\w^\star}, f_\out, f_{\w}$ defined by

	\begin{align}
		\vec{f}_{\w}(\bgamma ,\bLambda) &\equiv  \partial_\bgamma \log\(\mZ_{\w}(\bgamma,\bLambda)\) = \EE_{\rQ_{\w}} \[ \vec{w} \]\,, \label{appendix:definitions:update_generic:committee:prior:fw} \Spacecase 
		\partial_\bgamma \vec{f}_{\w} (\bgamma ,\bLambda) &\equiv  \EE_{\rQ_{\w}} \[\vec{w} \vec{w}^\intercal \] - \vec{f}_{\w}(\bgamma ,\bLambda) ^{\otimes 2} \label{appendix:definitions:update_generic:committee:prior:dfw}\,,\Spacecase
		\vec{f}_{\out} (y,\bomega, \bV) &\equiv \partial_\bomega \log \( \mZ_{\out}(y, \bomega, \bV) \) = \bV^{-1} \EE_{\rQ_{\out}} \[ \vec{z} - \bomega\]\,, \label{appendix:definitions:update_generic:committee:channel:fout} \Spacecase	
		\partial_{\bomega} \vec{f}_{\out} (y, \bomega, \bV) &\equiv \displaystyle \frac{\partial \vec{f}_{\out}(y,\bomega, \bV)}{\partial \bomega}\label{appendix:definitions:update_generic:committee:channel:dfout} \\
			&= \bV^{-1} \EE_{\rQ_{\out}} \[ \(\vec{z} - \bomega\)^{\otimes2} \]\bV^{-1} - \bV^{-1} - \vec{f}_{\out} (y,\bomega, \bV)^{\otimes2} \,.\notag
	\end{align}
Defining the natural variables $\bSigma = \mat{Q} - \mat{q}$ and $\hat{\bSigma}= \hat{\mat{Q}}+\hat{\mat{q}}$ they can reformulated as
\begin{align}
\begin{aligned}
	\hat{\mat{m}} &= \alpha \EE_{y, \bxi } \[ \mZ_{\out^\star} \times  \vec{f}_{\out^\star} \( y,  \mat{m} \mat{q}^{-1/2} \bxi, \brho_{\w^\star} - \mat{m}^\intercal \mat{q}^{-1}\mat{m}  \)  \right. \\
	& \left. \qquad \qquad \qquad \qquad \qquad \qquad \qquad \qquad \times \vec{f}_{\out} \( y,  \mat{q}^{1/2}\bxi, \bSigma \)^\intercal    \]\,, \\
	\hat{\mat{q}} &= \alpha \EE_{y, \bxi } \[ \mZ_{\out^\star} \(y,  \mat{m} \mat{q}^{-1/2} \bxi, \brho_{\w^\star} - \mat{m}^\intercal \mat{q}^{-1}\mat{m}    \)   \vec{f}_{\out} \(  y,  \mat{q}^{1/2}\bxi, \bSigma\)^{\otimes 2}    \]\,, \\
	\hat{\bSigma} &= - \alpha \EE_{y, \bxi } \[ \mZ_{\out^\star} \( y,  \mat{m} \mat{q}^{-1/2} \bxi, \brho_{\w^\star} - \mat{m}^\intercal \mat{q}^{-1}\mat{m}   \)  \right. \\
	& \left. \qquad \qquad\qquad \qquad \qquad \qquad \qquad \qquad \times \partial_\bomega \vec{f}_{\out} \(  y,  \mat{q}^{1/2}\bxi, \bSigma \)    \]\,,  \\
	\mat{m} &= \EE_{\bxi} \[ \mZ_{\w^\star} \times f_{\w^\star}\( \hat{\mat{m}}  \hat{\mat{q}}^{-1/2}  \bxi , \hat{\mat{m}}^\intercal \hat{\mat{q}}^{-1} \hat{\mat{m}}  \) \vec{f}_{\w} \(  \hat{\mat{q}}^{1/2}\bxi  , \hat{\bSigma} \)   \]\,, \\
	\mat{q} &= \EE_{\bxi} \[ \mZ_{\w^\star}\( \hat{\mat{m}}  \hat{\mat{q}}^{-1/2}  \bxi , \hat{\mat{m}}^\intercal \hat{\mat{q}}^{-1} \hat{\mat{m}}   \) \vec{f}_{\w} \( \hat{\mat{q}}^{1/2}\bxi  , \hat{\bSigma}\)^2   \]\,, \\
	\bSigma &= \EE_{\bxi} \[ \mZ_{\w^\star}\(\hat{\mat{m}}  \hat{\mat{q}}^{-1/2}  \bxi , \hat{\mat{m}}^\intercal \hat{\mat{q}}^{-1} \hat{\mat{m}}   \)  \partial_\bgamma \vec{f}_{\w} \(  \hat{\mat{q}}^{1/2}\bxi  , \hat{\bSigma}\) \] \,,
\end{aligned}
\label{appendix:se_equations_generic}
\end{align}
where we use the abusive notation $\EE_{y} = \int_\bbR \d y$.

\subsubsection{Bayes-optimal estimation}
Extremizing the Bayes-optimal free entropy eq.~\eqref{appendix:free_entropy_bayes}, we easily obtain the set of fixed point equations over the scalar parameters $\mat{q}_\bayes, \hat{\mat{q}}_\bayes$. It can be deduced from eq.~\eqref{appendix:se_equations_generic} using the Nishimori conditions $\vec{f}_{\w} = \vec{f}_{\w^\star}$, $\vec{f}_{\out}=\vec{f}_{\out^\star}$, $\mat{m}=\mat{q}, \bSigma = \brho_{\w^\star} - \mat{q}, \hat{\mat{m}}=\hat{\mat{q}}$ and $\hat{\bSigma}=\hat{\mat{q}}$ that lead to
\begin{align}
	\hat{\mat{q}}_\bayes &= \alpha \EE_{y, \bxi } \[ \mZ_{\out^\star} \( y,  \mat{q}_\bayes^{1/2}\bxi, \brho_{\w^\star} - \mat{q}_\bayes \)   \vec{f}_{\out^\star} \( y,  \mat{q}_\bayes^{1/2}\bxi, \brho_{\w^\star} - \mat{q}_\bayes \)^{\otimes 2}    \] \,, \nonumber \spacecase
	\mat{q}_\bayes &= \EE_{\bxi} \[ \mZ_{\w^\star}\( \hat{\mat{q}}_\bayes^{1/2}  \bxi , \hat{\mat{q}}_\bayes  \) \vec{f}_{\w^\star} \(  \hat{\mat{q}}_\bayes^{1/2}\bxi, \hat{\mat{q}}_\bayes\)^{\otimes 2}   \] \,.
\label{appendix:se_equations_generic:bayes}
\end{align}
